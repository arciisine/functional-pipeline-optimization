\section{Functional Programming and Pipelines}
Functional programming, and by implication functional isolation and purity lend themselves to code that is of higher quality \cite{langstudy14}, and by implication easier to maintain, and verify. Unfortunately the functional isolation generally results in poorer resource utilization by the computer. The increased resource cost versus code quality is usually an acceptable cost.

\subsection{Functional Pipelines}
It is a common practice, in functional programming, to handle processing of (looping) sequences by function invocation/recursion \cite{recurse16}.  Composing processing operators produces a \pipeline in which a list is transformed.  

\begin{minipage}{\linewidth}
\lstinputlisting[frame=single, caption={Pseudo-code for algorithm}, label={lst:pseudo}, language=js]{code/sample-pseudo.js}
\end{minipage} 

Given the algorithm defined in listing \ref{lst:pseudo}, there are a set of filters, transformations, aggregations that need to be applied in order to produce the desired output.  Each \pipelineoperator (\inlinecode{map}, \inlinecode{reduce}, \inlinecode{filter}, etc.) is comprised by two components.

\begin{enumerate} 
  \item The operator's agreed upon meaning.  
  \item The operator's use of its inputsinputs
\end{enumerate}

For example, \inlinecode{map} defines an operator that requires its inputs are a list, and a function that can transform every element of the list.  Once \inlinecode{map} is invoked with its inputs, it produces a new list with every element replaced with the transformation applied. All the operators have clear inputs and outputs per the \javascript specification. 

This abstraction comes with the cost of invoking a function per each element of the list and that is multiplied by number of operators in the pipeline. In addition to the cost of function invocation, there is also significant memory usage for \inlinecode{filter} and \inlinecode{map} operators as they produce an intermediate list (in addition to the original) to be used as input between operators (e.g. \inlinecode{map(filter(x, ...)...)}).  

\subsection{Functional Pipeline Operators: \javascript}

An implementation of this concept of \pipelines can be found in how \javascript models arrays.  The array class implements a super-set of the following functionality:

\begin{enumerate}
  \item \inlinecode{forEach(operation)}.  \inlinecode{operation} represents a generic function whose return value is ignored.  This method will invoke \inlinecode{operation} on every element of the array \cite{arrayforeach16}.

  \item \inlinecode{map(transform)}.  \inlinecode{transform} represents a mapping of the array contents.  This method will create a new array of identical size to the input, but with every element mapped through the \inlinecode{transform} function \cite{arraymap16}.

  \item \inlinecode{filter(predicate)}.  \inlinecode{predicate} represents a mapping of the array elements to a boolean value.  This method will return a new array in which every element that has a \inlinecode{predicate} invocation that returns \inlinecode{true} will be in the output \cite{arrayfilter16}.

  \item \inlinecode{reduce(accumulate, initial)}.  \inlinecode{accumulate} represents a function that receives both \inlinecode{accumulator} and an array element. Every time \inlinecode{accumulate} is invoked with \inlinecode{accumulator} and an array element, the output of the function is stored as a new value for \inlinecode{accumulate}.  This new value will be used on the next iteration or will be returned if it is the final iteration \cite{arrayreduce16}.

\end{enumerate}

Since \inlinecode{map} and \inlinecode{filter} both return arrays, this gives way to method chaining. Method chaining is a design pattern in which an operation on an object returns itself or an instance of the same type as the object. Visually, this produces a strong representation, as seen in listing \ref{lst:functional}, of the pseudo code.

\begin{minipage}{\linewidth}
\lstinputlisting[frame=single, caption={Functional implementation of findCommonWords}, label={lst:functional}, language=js]{code/sample-functional.js}
\end{minipage} 

\subsection{Performance Compensation}
A common optimization that is encouraged, throughout many programming languages, is to manually convert the functional traversal into a more traditional \inlinecode{for} loop \cite{iterperf09} \cite{iterperf10} \cite{iterperf11}.  This involves manually projecting the functional operator composition (and the companion functions) into standard procedural code.  While increasing performance and decreasing memory usage, the overall cost is tied to code maintainability and quality \cite{langstudy14}.  

The main trade-off of manual operation is pitting the programmer's desire for clean, testable code against the processor's need for tight loops and having as much of the code to be executed in a single function.  