\section{High Level}
The analysis framework attempts to mimic the assumptions and behavior of programmers when converting functional pipeline operators (\inlinecode{map}, \inlinecode{reduce}, \inlinecode{filter}, etc.) into \inlinecode{for} loop semantics. 

When we deal with the common cases and patterns, the algorithm is simple.  The true complexity arises in ensuring all places where this could be used do not violate correctness. There are many cases that are accounted for, but write-dependence is the only major caveat as discussed later.

The algorithm will run in two phases:
  \begin{enumerate}
    \item Source code analysis \secref{SourceCodeAnalysis}
    \item Runtime optimization \secref{RuntimeOptimization}
  \end{enumerate}

The first phase (as discussed in \secref{SourceCodeAnalysis}) will identify potential sites for optimization and will modify the candidate location accordingly.  The source code analysis only works on a single file at a time, to minimize the general complexity of the algorithm. In addition the single file perspective helps to keep the algorithm's assumptions as simple as possible, and defer decisions to runtime as often as possible.

The source code analysis will generate a new output file that will then trigger the runtime optimization.  At runtime the code will interrogate the candidate optimization, and if everything is in order, will execute a runtime optimization of the code. There are certain scenarios that disqualify optimization but these will not be known until runtime.  The end result involves compiling an alternate code path at runtime, and then executing the newly created function.  