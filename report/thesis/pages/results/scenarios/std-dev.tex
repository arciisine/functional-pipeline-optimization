\subsection{Standard Deviation}
A simple algorithm to compute the standard deviation of an array of numbers. It is computationally heavy.

For the most part, the optimized form outperforms the manual and standard functional form (figures \ref{fig:std-dev:1..100000..5000x2}, \ref{fig:std-dev:1..100000..5000x10}, \ref{fig:std-dev:10x1..100000..5000}, \ref{fig:std-dev:100x1..100000..5000}, \ref{fig:std-dev:1..100000..5000x100}).  What you can see though, specifically in the test with a small input size of two (Iterations vs Time (ns) with an Input Size of two), is that there is a cost to the overhead of the optimized code.  It is not very much, but the underlying \javascript engine is able to execute the operations on a two element array faster than the generated \inlinecode{for} loop.  

As with all optimizations of this nature, there will be edge cases that the optimization does not perform as fast.  This comes down to the trade-off analysis of determining if the increased performance is worth the cost of code maintenance, in addition to looking at the other cases where the optimized form does outperform.