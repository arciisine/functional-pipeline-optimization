\section{Setup}

For the majority of our testing, we used NodeJS, and specifically v8 (~5.3).  NodeJS provides a high performance timer that provides nano second resolution.  Given the speed on many of the algorithms, the metrics (average, min, max) all register in the sub-millisecond region.  

Different engines will invariably produce different results on the scenarios we are testing.  The goal of the optimization is still to perform what a programmer would do by hand, and so in that vein, the underlying engine is generally unimportant.  There may be some additional performance from the runtime optimization process, because we are generating that at runtime, and it is possible to tailor the output to the specific environment the code is being run in.

Additionally, ChromeOS, NodeJs, and Android provide a sufficiently large ecosystem running v8 that any findings are significant regardless of other JavaScript environments.

