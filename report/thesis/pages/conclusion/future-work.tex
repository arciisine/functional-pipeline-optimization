\section{Future Work}

The concept of composing functional operations, at runtime or compile time, has implications beyond \javascript and \pipelines. Working with a language that is statically typed would remove a lot of the guess work from the process in general, and would allow for more consistent and rigorous optimizations.  Taking the optimization out of the \javascript code and moving it into the engine would open up many more avenues for optimizations and would also minimize the duplicated effort in parsing and compiling the \javascript code at runtime.

Another area of optimization would be moving the \algorithm towards stream-based programming.  Streams (generators) are powerful constructs, and stream transformation/filtering/reduction is useful.  RxJS \cite{rxjs16} is a stream based framework for \javascript that is gaining in popularity. It is an implementation of the general design pattern called Reactive Extensions \cite{rxio16}. The core of the framework are streams(Observable) and this same concept is poised to gain first class status within the \javascript language \cite{observablejs16}. The ability to create new streams, and to compile the composition to a single function versus a nested of set of function invocations could provide a significant boost to the general performance of the framework and all that rely upon it. 

Beyond all of this, there are more general optimizations that could be made, by the nature of recompiling the code at runtime, and specifically merging multiple bodies of code together into a single function.  The \veight \javascript engine is already great at optimizing where it can, but even just reducing the amount of code it has to evaluate (constant folding, algebraic simplifications, dead code elimination, etc. would be useful enhancements).