\begin{romanpages}      % roman-numbered pages 

\TitlePage 

\begin{abstract} 
There has always been a disconnect between how humans organize computer code, and how computers execute it.  Functional programming is a paradigm that increases code maintainability and testability, but generally results in poorer resource utilization by the computer.  The increased resource cost versus code quality is usually an acceptable cost. However, code that is executed many times over, would benefit from optimization as the cost of the introduced abstractions are amplified. It is possible to dynamically translate a subset of the functional paradigm into a form that is optimized for computer execution. This hybrid ultimately provides the best of both worlds, allowing the human to write and manage code in a manner that is familiar and easier to reason about, and for the final output to be in a form that executes with higher efficiency. 
\end{abstract}

\begin{acknowledgments}
...
\end{acknowledgments}

\tableofcontents
\listoffigures
\listoftables

\printnomenclature[0.5in] %used for the List of Abbreviations
\end{romanpages}        % All done with roman-numbered pages


\normalem       % Make italics the default for \em