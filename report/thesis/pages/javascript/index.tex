\section{JavaScript}

JavaScript (officially ECMAScript) is a high-level, dynamic, untyped, programming language. Although there are strong outward similarities between JavaScript and Java, including language name, syntax, and respective standard libraries, the two are distinct languages and differ greatly in their design. JavaScript was influenced by programming languages such as Self and Scheme. JavaScript has been traditionally implemented as an interpreted language, but more recent implementations perform just-in-time compilation. \cite{javascript16}

JavaScript is a very powerful language, supporting functional programming, as well as procedural and object oriented paradigms.  Additionally, the ability to compile code at runtime makes it powerful (and dangerous) for expressing complex operations or computations. \cite{eval16} 

\subsection{ES2015}
ES2015 is the 2015 revision of the JavaScript language. This revision introduces new grammar constructs, which in turn have increased the implementation complexity \cite{arrowfn16} \cite{let16} \cite{destructuring16}.  Here are  some of the features introduced in ES2015:

  \begin{enumerate}
    \item \textbf{Let, Const} These concepts allow for local variable declaration and local constant declaration.  The main change with this feature is that JavaScript now supports block level scoping as opposed to the previous state of only scoping at the functional or global level \cite{let16} \cite{const16}.  This allows for clearer code (and fewer bugs) but requires a different algorithm for calculating variable scopes.  By contrast \inlinecode{var} scopes at the function level, regardless of location within the function (or nested blocks).  This is known as variable-hoisting as all variables are assumed to be available at the first point of the function.
    \item \textbf{Fat Arrow Functions} This is generally a short-hand for anonymous functions with one major caveat.  The fat arrow function (\inlinecode{x => x * this.y}) treats the \inlinecode{this} variable differently \cite{arrowfn16}.  Generally most functions have their own context, referenced by a variable \inlinecode{this}.  Since functions and classes are nearly identical in JavaScript, \inlinecode{this} is a reference to the instance.  Fat arrow functions refuse to create a \inlinecode{this} for reference, and instead defer to the enclosing function scope for the reference of \inlinecode{this}. Again, variable declaration and tracing are more complicated due to the additional scoping methodology.
    \item \textbf{Object/Array Destructuring}  This concept allows a programmer to unpack the contents of an array or an object literal into its constituent pieces, at the time of assignment \cite{destructuring16}.  This includes variable declarations as well as function parameters.  While this is enhances expressive the power of the language, it also increases the complexity of variable tracking.  
  \end{enumerate}

The reason these features are highlighted is that they all play a role in the final solution.  This is also to highlight that as the language grows and changes, the optimization algorithm may need to be modified to stay consistent. 