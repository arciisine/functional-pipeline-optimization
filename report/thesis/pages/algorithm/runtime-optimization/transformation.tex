\subsection{Transformation}

The transformation process only deals with a single operator at a time.  Given something as simple as $filter(x => x > 2)$.  This could be translated into $if (x <= 2) continue$ assuming the code is running a for-loop.

The goal of the transformation process is do five things:
\begin{enumerate}
  \item \textbf{Verify reference (non-literal) functions are valid}
    The main verification piece is to verify that the function is not expecting access to the special
    form of the majority of the array operators.  $map$, $reduce$, $filter$, $forEach$, $some$, $find$, etc.  all have an optional last argument that provides read only access to the intermediate array.  Since we are skipping creating the intermediate array intentionally (and almost no one uses this form), any use of this special form require that we bail on the entire chain. Theoretically we can do partial chains, around the offending element, but that would require moving even more of the source code analysis process into the compilation process.  Building an intermediate array is not acceptable  as we are modifying the order of operations and the special form array is meant to represent the entire array in the previous transformation.

    This points to a larger issue, in which any reference is made to the intermediate array, generally disqualifies a chain from optimization.
  
  \item \textbf{Determine for references, if a function can be inlined}
    For references of function parameters, we already know this form cannot be inlined.  For static references, we need to do code analysis and look for any closed variables.  A pure function (without closed variables) can be inlined.  If closed variables exist, the static reference must be treated as a dynamic reference.
  \item \textbf{Determine if a function has need of a positional counter}
    From this point, we inspect the function parameters and look to see if it has a parameter for the index.  If it does not, we skip intializing and incrementing the position for the specific transformation.  If we cannot read the function (e.g. it is a native function) or if it does reference the position parameter, then we will initialize and increment.
  \item \textbf{Generate the partial AST}
    Once all the above is done we generate one of two ASTs. Given a filter predicate of $isEven$ we can generate an inline call $if (!(x \mod 2 == 0)) continue$ or we will generate an invocation call $if (!isEven.call(context, x)) continue$.  Both are superior to creating and freeing intermediate arrays, but the inline call is still best.
  \item \textbf{Rewrite Local Variables and Parameters}
    Once the AST is provided, we rewrite all function parameters and references to $this$ to allow for multiple transformations to reside within the same for loop.  Any variables with the same name would collide and cause problems.  
\end{enumerate}