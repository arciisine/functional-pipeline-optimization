\subsection{Compilation}\label{Compilation}
When constructing the resultant optimized code, we need to initialize the environment the code will be running in.  This involves aliasing in closed variables, exposing all the input parameters for the individual transformations (e.g. this context, or the initial value for an accumulator, etc.).  The goal here is to provide all the relevant structure needed for runtime.  

In addition to the initialization, the compilation process also provides the $for-loop$ to iterate through all the data elements as well as mapping the closed variables into the final output, for reassignment once the function finishes.  A simple example of a transformation is found in listing \ref{lst:simplefunctionalcompiled}.

\begin{minipage}{\linewidth}
\lstinputlisting[frame=single, caption={Simple Functional Compiled}, label={lst:simplefunctionalcompiled}, language=JavaScript]{code/simple-functional-compiled.js}
\end{minipage}