\subsection{Invocation Form}
The next step is to process the candidate optimization into the final invocation form.  This massages the the call chain into a final form that allows for maximal performance at runtime.  The entire goal of the invocation form is to minimize any overhead at runtime.  The reason for this, is that the overhead for execution of the optimized form, assuming chain target is an array, will have significant performance impact smaller arrays.  For the transformation to be generally useful it needs to perform well on arrays of size $0$ as well as arrays of size $100000$.  

\begin{minipage}{\linewidth}
\lstinputlisting[frame=single, language=JavaScript]{code/simple-functional-final.js}
\end{minipage}

Given the goal of minimizing the final form's computation, the chain is reorganized into predetermined arrays that can be handed piecemeal to different functions without the need for any memory allocations, string concatenations or function invocations.

When invoking an optimized code path, the only data that is needed is the $target$, $context$, $closed$ and $closure assignment$.  These variables are utilized by the optimized code path, in the expressed form. 

The final form also needs to accomodate when the $target$ is not an array, and needs to be invoked in a manner equivalent to the original form.  \\

\begin{minipage}{\linewidth}
\lstinputlisting[frame=single, language=JavaScript]{code/manual-invocation.js}
\end{minipage}

All that occurs is each chain operation is invoked as a member of chain target (with the appropriate input), and that is stored as the new chain target.  This is functionally equivalent, and generally as performant as the original form. %Needs proof/verification

\subsubsection{Closure Analysis}
Within the final form you can also see that we analyze closed variables to be able to handle reading and writing of these variables within the optimized code path. Because the optimized code path is not in the same scope as the candidate optimization, we lose access to the closed variables.  We need to calculate all the closed variables that we can, and handle them accordingly.  Due to the fact that we are only looking at one file at a time, we are only able to analyze closed variables for inline functions, and everything else can only be handled at runtime.  
\paragraph{Parameters}
Another piece of the invocation form is knowledge of which parameters are passed in as function parameters versus which are inline or are globally defined.

The issue here is that we should assume that any parameter passed in will not be static and thus we have to treat as simply as possible, due to the fact that even analysis of the function itself may take longer than running the code in the original form.  And so we assume the worst case scenario for these functions and limit the optimization to merely invoking the parameter as opposed to attempting to inline the function at runtime.

Hopefully this highlights odd partitioning the problem, and shows the intricacy of the interplay between source code analysis and the runtime optimization.  