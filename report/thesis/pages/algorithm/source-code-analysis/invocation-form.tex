\subsection{Invocation Form}
The next step is to process the candidate optimization into the final invocation form.  This massages the functional pipeline into a final form that allows for maximal performance at runtime by minimizing any runtime overhead.  The overhead for execution of the optimized form, assuming functional pipeline target is an array, will have significant performance impact smaller arrays, and needs to be mitigated as much as possible.  For the transformation to be generally useful it needs to perform well on arrays of size $0$ as well as arrays of size $100000$.  

\begin{minipage}{\linewidth}
\lstinputlisting[frame=single, caption={Simple Functional, Final Form}, label={lst:simplefunctionalfinal}, language=js]{code/simple-functional-final.js}
\end{minipage}

Given the goal of minimizing the final form's computation, the functional pipeline is restructured into predetermined arrays that can be handed piecemeal to the runtime optimizer without the need for any memory allocations, string concatenations or function invocations.

When invoking an optimized code path, the only data that is needed is the \inlinecode{target}, \inlinecode{context}, \inlinecode{closed} and \inlinecode{closure assignment}.  These variables are utilized by the optimized code path in the expressed form. 

The final form also needs to accommodate when the \inlinecode{target} is not an array, and needs to be invoked in a manner equivalent to the original form.  \\

\begin{minipage}{\linewidth}
\lstinputlisting[frame=single, caption={Manual invocation}, label={lst:manualinvocation}, language=js]{code/manual-invocation.js}
\end{minipage}

In the code listing \ref{lst:manualinvocation}, each pipeline stage is invoked as a member of the pipeline target (with the appropriate input), and that is stored as the new pipeline target for the next pipeline stage.  This is functionally equivalent, and should perform comparably to the original form. 

\subsubsection{Closure Analysis}
Within the final form you can also see that we analyze closed variables to be able to handle reading and writing of these variables within the optimized code path. Because the optimized code path is not in the same scope as the candidate optimization, we lose access to the closed variables.  We need to calculate all the closed variables that we can, and handle them accordingly.  Due to the fact that we are only looking at one file at a time, we are only able to analyze closed variables for inline functions, and everything else can only be handled at runtime.  
\paragraph{Parameters}
Another piece of the invocation form is knowledge of which parameters are passed in as function parameters versus which are inline or are globally defined.

The issue here is that we should assume that any parameter passed in will not be static and thus we have to treat as simply as possible, due to the fact that even analysis of the function itself may take longer than running the code in the original form.  And so we assume the worst case scenario for these functions and limit the optimization to merely invoking the parameter as opposed to attempting to inline the function at runtime.

The intricacy of the interplay between source code analysis and the runtime optimization is non-trivial and great care is needed to balance the needs of the various phases with overall performance.  