\subsection{Identifying Functional Pipelines}

As noted before, method chaining is a common paradigm in \javascript for creating functional pipelines over a list of elements.  The final composed entity is identified as a functional pipeline, and that is what we are looking for when optimizing.  Essentially, we are looking for a specific call expression, that has an array as the call target. \\

\begin{minipage}{\linewidth}
\lstinputlisting[frame=single, label={lst:simplefunctional}, caption={Simple Functional Example}, language=js]{code/simple-functional.js}
\end{minipage}

In listing \ref{lst:simplefunctional}, the \inlinecode{map} call expression on line 4 is the the beginning of our pipeline, and the pipeline terminates on line 11 with a \inlinecode{map} call expression.  It is easy to recognize this visually, but the source code analysis can only be so certain.  In general, it finds a potential call expression, which is defined as a call expression in which the chained operator is a well known identifier (e.g. \inlinecode{map}, \inlinecode{filter}, \inlinecode{reduce}, etc.). Once the base of the functional pipeline is determined, the code will follow the chain until there are no further call expressions or an incompatible call expression is found (e.g. \inlinecode{sort}, \inlinecode{reverse}, etc.)

At this point the candidate optimization is identified.  It is still not know if the source of the functional pipeline is truly an array or if it can alternate types at runtime, and this is why the functional pipeline  is identified as a candidate for optimization. 