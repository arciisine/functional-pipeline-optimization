\subsection{Write Dependences}

The optimization's entire structure is centered around altering the order of execution, specifically to avoid any intermediate storage of results. The issue of write dependence arises from the change in order of operation in visiting each element of the sequence.  In the functional pipeline form, each pipeline stage must finish completely before the next stage can start.  In the procedural form, each pipeline stage is applied in succession on each sequence element.  Again, this is an identical problem a programmer would face if they were to manually convert the candidate optimization into a procedural form by hand.  Additionally, since property accesses can actually result in writing values, and methods can be rewritten at runtime, determining when a write actually occurs is fairly complicated. 

Ultimately this leads us to the point that we will ignore the concept of write dependences as they cannot be proven except at runtime (if even then).  We will behave in a manner that is consistent with a hand generated optimization.  If we were to allow optimizations only on literal function parameters (numbers, strings, boolean values) with the rule that no properties can be accessed, we would have a higher degree of certainty that we will not have any write dependences.  Ultimately this would make the optimization far less useful, as it would apply in only the simplest cases.  