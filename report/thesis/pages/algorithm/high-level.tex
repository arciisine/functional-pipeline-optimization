\section{High Level}
In general, the algorithm attempts to mimic the assumptions and behavior of programmers when converting functional list operations ($map$, $reduce$, $filter$, etc.) into $for-loop$ semantics.

When dealing with the common cases and patterns, the algorithm is simple.  The real complexity arises in ensure all places where this could be used do not violate correctness.  There is still the real issue of write-dependence between operations as each iteration runs through all operations instead of running an operation through all elements at once.

The algorithm will run in two phases:
  \begin{enumerate}
    \item Source code analysis
    \item Runtime optimization
  \end{enumerate}

The first phase (Source code analysis) will identify potential sites for optimization and will modify the candidate location accordingly.  The source code analysis only works on a single file at a time, to minimize the general complexity and performance of the algorithm. In addition the single file perspective helps to keep the algorithm's assumptions as simplistic as possible, and defer as much as possible to the runtime.

The source code analysis will feed into the runtime optimization.  At runtime the code will interrogate the candidate optimization, and if everything is in order, will execute a runtime optimization of the code.  This involves compiling an alternate code path at runtime, and then executing the alterante code path.  This also involves potentially reading the source of some functions at runtime to utilize the source code to help build the final optimization.
