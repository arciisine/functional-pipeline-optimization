  \subsection{Tagging Functions}
  One of the key pieces needed for performance is uniquely identifying each callsite, and precomputing as much as possible for each unique operation.  Because the keys require some certain level of inforamtion, computing them at runtime can be costly.  

  \begin{enumerate}
    \item Static/Inline
      \begin{enumerate} 
        \item These functions can have their key computed at compile time since they are static
        \item For a functional list expression, we can coalesce all neighboring static keys to minimize computation
        \item IF all keys are static for a functional list expression, we can reduce it to a single unique key of minimal size
        \item If a static form (not a function parameter) is found to have closed variables at runtime we need to compute it's key. 
          \begin{enumerate}
            \item This is the worst case scenario as it requires additional computation for the key which is overhead for every execution.  If the number of iterations doesn't overcome the cost of key computation, we are losing.
          \end{enumerate}
      \end{enumerate}
    \item Dynamic
      \begin{enumerate} 
        \item If dynamic is known at compile-time (and using invocation form) we can also precompute a unique key since it should always produce the same form.
      \end{enumerate}
  \end{enumerate}