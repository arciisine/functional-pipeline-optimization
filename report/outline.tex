\documentclass{article}
\usepackage[utf8]{inputenc}
\usepackage[table]{xcolor}
\usepackage{tikz}
\usepackage{amsmath}
\usepackage{amsfonts}
\usepackage{amssymb}
\usepackage{amsthm}
\usepackage{stmaryrd}
\usepackage{array}
\usepackage{enumitem}
\usepackage{verbatim}
\usepackage{color}
\usepackage{listings}

\usetikzlibrary{arrows,positioning,calc}
\usetikzlibrary{graphs,arrows.meta}

\setlist[itemize]{leftmargin=*}
\setlist[enumerate]{leftmargin=*}

\definecolor{lightgray}{rgb}{.9,.9,.9}
\definecolor{darkgray}{rgb}{.4,.4,.4}
\definecolor{darkblue}{rgb}{.4,.4,0}
\definecolor{purple}{rgb}{0.65, 0.12, 0.82}

\lstdefinelanguage{JavaScript}{
  keywords={typeof, new, let, const, true, false, catch, function, return, null, catch, switch, var, if, in, while, do, else, case, break},
  keywordstyle=\color{blue}\bfseries,
  ndkeywords={class, export, boolean, throw, implements, import, this, filter, map, reduce, foreach, some, find, WRAP, EXEC, TAG},
  ndkeywordstyle=\color{darkblue}\bfseries,
  identifierstyle=\color{black},
  sensitive=false,
  comment=[l]{//},
  morecomment=[s]{/*}{*/},
  commentstyle=\color{purple}\ttfamily,
  stringstyle=\color{red}\ttfamily,
  morestring=[b]',
  morestring=[b]"
}

\lstset{
   language=JavaScript,
   backgroundcolor=\color{lightgray},
   extendedchars=true,
   basicstyle=\footnotesize\ttfamily,
   showstringspaces=false,
   showspaces=false,
   numbers=left,
   numberstyle=\footnotesize,
   numbersep=9pt,
   tabsize=2,
   breaklines=true,
   showtabs=false,
   captionpos=b
}


\usepackage[margin=1in]{geometry}

\title{Thesis Outline}
\author{Timothy Soehnlin}
\date{\today}

\begin{document}

\section{Goals}
Provide optimized code while retaining ability to utilize functional list paradigms.

\section{Algorithm}
At a high level, the algorithm will transform a series of standard JavaScript $Array$ operators ($filter$, $map$, 
$reduce$, $forEach$, $some$, $every$) into a standard for loop notation.  

The full algorithm can be broken into the following process:
\begin{enumerate}
  \item Transpile source files to look for possible transformation patterns
  \begin{enumerate}
    \item Look for every possible indication that Array list operators are being used
    \begin{enumerate}
      \item Only $map$ and $filte$r reliably return new Arrays, whereas $some$, 
        $forEach$ and $every$ reliably return non-arrays.  $reduce$ can return 
        any value (and so we treat it as unknown)    
    \end{enumerate}
    \item When a potential Array operator site is found, modify source to
    \begin{enumerate} 
      \item	Wrap the Array expression with a $WRAP$ invocation, to provide an Array surrogate that can collect the operators
      \item	Wrap every function used by the operators with a $TAG$ invocation.  This provides additional metadata used for the compilation process
      \item	Wrap the potential Array expression and the subsequent operators with an $EXEC$ invocation to allow for compilation and execution of the collected operators
    \end{enumerate}    
  \end{enumerate}

  Sample Transpilation input  
  \lstinputlisting[frame=single, language=JavaScript]{sample_target.js}

  Sample Transpilation output  
  \lstinputlisting[frame=single, language=JavaScript]{sample_target_transpiled.js}

  \item At execution time, the functions $WRAP$, $TAG$, and $EXEC$ will perform all the necessary work to
  produce and execute the optimized code
    \begin{enumerate}
    \item When $WRAP$ is invoked, and the argument is an array, will produce a wrapper object that will collect 
      all the tagged operations ($filter$, $map$, $reduce$, etc.).  
    \item When $EXEC$ is invoked, if the input is a wrapper object, it will compile and execute the code.      
    \item When $WRAP$ is invoked with a non-array or $EXEC$ is invoked with a non-wrapper object, 
      it will return the input as is.  This allows for runtime type detection, and will not fail if 
      types are not-known at compile-time.  
    \end{enumerate}  
\end{enumerate}
\end{document}